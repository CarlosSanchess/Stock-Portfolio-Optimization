\documentclass{beamer}
\usepackage{graphicx} % For including images (if needed)
\usepackage{hyperref} % For adding clickable links to references
\usepackage{cite} % For managing references and citations

\title{Portfolio Optimization Using Metaheuristic Algorithms}
\subtitle{Artificial Intelligence Report}
\author{
    Carlos Sanches, up202107694 \\
    João Rebelo, up202107209 \\
    Gonçalo Sousa, up202207320
}

\setbeamertemplate{date}{}

\begin{document}

% Title slide
\begin{frame}
    \titlepage
\end{frame}

% Slide 1: Problem Specification
\begin{frame}
    \frametitle{Problem Specification}
    \textbf{Portfolio Optimization Tool}

    \textbf{Objective:} Develop a robust tool to optimize investment portfolios using metaheuristic algorithms.

    \textbf{Key Features:}
    \begin{itemize}
        \item Determine optimal asset allocation to maximize returns while minimizing risk.
        \item Support multiple optimization algorithms (Hill Climbing, Simulated Annealing, Tabu Search, Genetic Algorithm).
        \item Allow customizable constraints (max/min weights, risk aversion, short selling).
        \item Provide interactive visualization of optimization results.
    \end{itemize}

    \textbf{Expected Outcome:} A user-friendly application that produces efficient portfolios based on historical data from Australian Stock Market (2000-2020).
\end{frame}

% Slide 2: Related Work
\begin{frame}
    \frametitle{Related Work}
    \textbf{Literature Review}

    \textbf{Modern Portfolio Theory (MPT):}
    \begin{itemize}
        \item Markowitz, H. (1952). "Portfolio Selection." Journal of Finance, 7(1), 77-91.
        \item Introduced mean-variance optimization framework for portfolio construction.
    \end{itemize}

    \textbf{Metaheuristic Approaches:}
    \begin{itemize}
        \item Chang, T.J., et al. (2000). "Heuristics for cardinality constrained portfolio optimisation." Computers \& Operations Research, 27(13), 1271-1302.
        \item Cura, T. (2009). "Particle swarm optimization approach to portfolio optimization." Nonlinear Analysis: Real World Applications, 10(4), 2396-2406.
    \end{itemize}

    \textbf{Dataset and Software implementation}
    \begin{itemize}
        \item Australian Historical Stock Prices Dataset and Notebooks.
        \item \url{https://www.kaggle.com/datasets/ashbellett/australian-historical-stock-prices/data}
    \end{itemize}
\end{frame}

% Slide 3: Problem Formulation
\begin{frame}
    \frametitle{Problem Formulation}
    \textbf{Portfolio Optimization as an Optimization Problem}

    \textbf{Solution Representation:}
    \begin{itemize}
        \item Vector $w = (w_1, w_2, ..., w_n)$ representing portfolio weights for $n$ assets.
        \item Constraint: $\sum_{i=1}^{n} w_i = 1$.
    \end{itemize}

    \textbf{Objective Function:}
    \[
    U(w) = E[R_p] - \lambda \cdot \sigma_p^2
    \]
    Where:
    \begin{itemize}
        \item $E[R_p] = \sum_{i=1}^{n} w_i \cdot \mu_i$ (expected portfolio return).
        \item $\sigma_p^2 = w^T \Sigma w$ (portfolio variance).
        \item $\lambda$ is risk aversion parameter.
    \end{itemize}
    Maximize $U(w)$ or Sharpe ratio $\frac{E[R_p]}{\sigma_p}$.

    \textbf{Constraints:}
    \begin{itemize}
        \item Weight bounds: $0 \leq w_i \leq max\_weight$ (without short selling).
        \item With short selling: $-1 \leq w_i \leq 1$.
        \item Minimum return: $E[R_p] \geq min\_return$.
        \item Maximum risk: $\sigma_p \leq max\_risk$.
    \end{itemize}
\end{frame}

% Slide 4: Algorithms and Operators
\begin{frame}
    \frametitle{Algorithms and Operators}
    \textbf{Metaheuristic Approach}

    \textbf{Hill Climbing:}
    \begin{itemize}
        \item Neighborhood function: Random perturbation of weights with step size.
        \item Restart mechanism to avoid local optima.
    \end{itemize}

    \textbf{Simulated Annealing:}
    \begin{itemize}
        \item Temperature schedule: Initial\_temp $\times$ Cooling\_rate$^{iteration}$.
        \item Acceptance probability: $e^{\frac{\Delta U}{T}}$ for negative utility changes.
    \end{itemize}

    \textbf{Tabu Search:}
    \begin{itemize}
        \item Tabu list: Recent weight configurations to avoid cycling.
        \item Aspiration criterion: Accept tabu moves if they improve best solution.
    \end{itemize}

    \textbf{Genetic Algorithm:}
    \begin{itemize}
        \item Crossover: Weighted average of parent solutions.
        \item Mutation: Random perturbation with mutation rate.
        \item Selection: Tournament selection with elitism.
    \end{itemize}
\end{frame}
%Slide 5: Results Overview and thou 

\begin{frame}{Our Analysis}

    
    \textbf{Common test scenarios for all Algorithms:}
    \begin{itemize}
        \item TODO.
        \item TODO.
        \item TODO.
    \end{itemize}
    \vspace{4 pt}
    \textbf{Specific Analysis:}
    \begin{itemize}
        \item \textbf{Hill Climbing:}
        \item \textbf{Simulated Annealing:}
        \item \textbf{Tabu Search:}Aspiration criteria and Tabu
        Tenure impact.
        \item \textbf{Genetic Algorithm:}

    \end{itemize}
    

\end{frame}

\begin{frame}{Hill Climbing}

    
    

\end{frame}


\begin{frame}{Simulated Anealing}

    
    

\end{frame}

\begin{frame}{Tabu Search}

    
    

\end{frame}
% Slide 5: Implementation Details
\begin{frame}
    \frametitle{Implementation Details}

    \textbf{Programming Language:} Python 3.

    \textbf{Development Environment:}
    \begin{itemize}
        \item GUI: Tkinter framework.
        \item Visualization: Matplotlib for interactive plots.
    \end{itemize}

    \textbf{Key Data Structures:}
    \begin{itemize}
        \item pandas.DataFrame for historical price data and returns calculation.
        \item NumPy arrays for efficient matrix operations.
    \end{itemize}

    \textbf{Current Progress:}
    \begin{itemize}
        \item Complete GUI implementation with data loading capabilities.
        \item Hill Climbing algorithm implementation.
        \item Performance visualization (portfolio pie chart, allocation details risk rate, annual return etc).
        \item Data preprocessing and portfolio metrics calculation.
    \end{itemize}

    \textbf{Next Steps:}
    \begin{itemize}
        \item Implement remaining optimization algorithms.
        \item Add comparative performance evaluation.
        \item Improve data visualization and insights about the portfolio.
    \end{itemize}
\end{frame}

\end{document}